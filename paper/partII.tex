\documentclass[letterpaper,final,12pt,reqno]{amsart}

\usepackage[total={6.3in,9.2in},top=1.1in,left=1.1in]{geometry}

\usepackage{times,bm,bbm,empheq,fancyvrb,graphicx}
\usepackage[dvipsnames]{xcolor}
\usepackage{longtable}
\usepackage{booktabs}

\usepackage{tikz}
\usetikzlibrary{decorations.pathreplacing}

\usepackage[kw]{pseudo}
\pseudoset{left-margin=15mm,topsep=5mm,idfont=\texttt}

% hyperref should be the last package we load
\usepackage[pdftex,
colorlinks=true,
plainpages=false, % only if colorlinks=true
linkcolor=blue,   % ...
citecolor=Red,    % ...
urlcolor=black    % ...
]{hyperref}

\renewcommand{\baselinestretch}{1.05}

\newtheoremstyle{claim}% name
  {5pt}% space above
  {5pt}% space below
  {\itshape}% body font
  {}% indent amount
  {\itshape}% theorem head font
  {.}% punctuation after theorem head
  {.5em}% space after theorem head
  {\thmname{#1}\thmnumber{ #2}\thmnote{ (#3)}}% theorem head spec
\theoremstyle{claim}
\newtheorem{theorem}{Theorem}
\newtheorem{lemma}{Lemma}

\newcommand{\eps}{\epsilon}
\newcommand{\RR}{\mathbb{R}}

\newcommand{\grad}{\nabla}
\newcommand{\Div}{\nabla\cdot}
\newcommand{\trace}{\operatorname{tr}}

\newcommand{\hbn}{\hat{\mathbf{n}}}

\newcommand{\bb}{\mathbf{b}}
\newcommand{\be}{\mathbf{e}}
\newcommand{\bbf}{\mathbf{f}}
\newcommand{\bg}{\mathbf{g}}
\newcommand{\bn}{\mathbf{n}}
\newcommand{\br}{\mathbf{r}}
\newcommand{\bu}{\mathbf{u}}
\newcommand{\bv}{\mathbf{v}}
\newcommand{\bw}{\mathbf{w}}
\newcommand{\bx}{\mathbf{x}}

\newcommand{\bF}{\mathbf{F}}
\newcommand{\bV}{\mathbf{V}}
\newcommand{\bX}{\mathbf{X}}

\newcommand{\bxi}{\bm{\xi}}

\newcommand{\bzero}{\bm{0}}

\newcommand{\rhoi}{\rho_{\text{i}}}

\newcommand{\ip}[2]{\left<#1,#2\right>}

\newcommand{\mR}{R^{\bm{\oplus}}}
\newcommand{\iR}{R^{\bullet}}

\newcommand{\pp}{{\text{p}}}
\newcommand{\qq}{{\text{q}}}
\newcommand{\rr}{{\text{r}}}

% numbering
\setcounter{tocdepth}{3}
\makeatletter
\def\l@subsection{\@tocline{2}{0pt}{4pc}{5pc}{}}
\makeatother

\numberwithin{equation}{section}
\numberwithin{figure}{section}
\numberwithin{table}{section}
\numberwithin{theorem}{section}


\begin{document}
\title[Geometric multigrid for glacier modeling II]{Geometric multigrid for glacier modeling II: \\ Glacier geometry from Stokes dynamics}

\author{Ed Bueler}

\begin{abstract} FIXME MCD for evolving-geometry with Glen-Stokes dynamics; Schur-complement or Vanka Newton-multigrid for the dynamics problem
\end{abstract}

\maketitle

\tableofcontents

\thispagestyle{empty}
%\bigskip

\section{Introduction} \label{sec:intro}

FIXME MCD = multilevel constraint decomposition, a multigrid \cite{Trottenbergetal2001} method basically by \cite{Tai2003} for obstacle problems; see part I \cite{Bueler2022partI};  obstacle problem view first extended to Stokes by \cite{WirbelJarosch2020}


\section{MCD for the Glen-Stokes geometry problem} \label{sec:stokesgeometry}

FIXME state steady continuum strong form \cite[Chapter 1]{FowlerNg2021} and \cite{GreveBlatter2009}; numerical solutions by various methods e.g.~\cite{Jouvetetal2008}

FIXME weak form

FIXME smoother uses Firedrake \cite{Alnaesetal2014,Rathgeberetal2016} with extruded meshes \cite{Gibsonetal2019,McRaeetal2016} and PETSc \cite{Balayetal2020,Bueler2021} to solve the dynamics problem and thereby evaluate the residual, and Stokeslets for Jacobian

FIXME time-dependent runs


\section{Multigrid for the (fixed-geometry) Glen-Stokes dynamics problem} \label{sec:stokesdynamics}

FIXME weak form; well-posed by \cite{JouvetRappaz2011}; various numerical solutions e.g.~\cite{Lengetal2012}

FIXME multigrid already used for Blatter-Pattyn dynamics \cite{BrownSmithAhmadia2013}; for hybrid dynamics \cite{Jouvetetal2013,JouvetGraeser2013}; and for Stokes dynamics \cite{IsaacStadlerGhattas2015} and \cite{Tuminaroetal2016} using AMG

FIXME we use Schur complement \cite{Bueler2021,Elmanetal2014} and compare it to Vanka monolithic smoother \cite{Farrelletal2019}


\small

\bigskip
\bibliography{partII}
\bibliographystyle{siam}

\end{document}
