\documentclass[letterpaper,final,12pt,reqno]{amsart}

\usepackage[total={6.3in,9.2in},top=1.1in,left=1.1in]{geometry}

\usepackage{times,bm,bbm,empheq,fancyvrb,graphicx}
\usepackage[dvipsnames]{xcolor}
\usepackage{tikz}
\usetikzlibrary{decorations.pathreplacing}

% hyperref should be the last package we load
\usepackage[pdftex,
colorlinks=true,
plainpages=false, % only if colorlinks=true
linkcolor=blue,   % ...
citecolor=Red,    % ...
urlcolor=black    % ...
]{hyperref}

\renewcommand{\baselinestretch}{1.05}

\newtheorem{lemma}{Lemma}

\newcommand{\Matlab}{\textsc{Matlab}\xspace}
\newcommand{\eps}{\epsilon}
\newcommand{\RR}{\mathbb{R}}

\newcommand{\grad}{\nabla}
\newcommand{\Div}{\nabla\cdot}
\newcommand{\trace}{\operatorname{tr}}

\newcommand{\hbn}{\hat{\mathbf{n}}}

\newcommand{\bb}{\mathbf{b}}
\newcommand{\bbf}{\mathbf{f}}
\newcommand{\bg}{\mathbf{g}}
\newcommand{\bn}{\mathbf{n}}
\newcommand{\bu}{\mathbf{u}}
\newcommand{\bv}{\mathbf{v}}
\newcommand{\bw}{\mathbf{w}}
\newcommand{\bx}{\mathbf{x}}

\newcommand{\bV}{\mathbf{V}}
\newcommand{\bX}{\mathbf{X}}

\newcommand{\bxi}{\bm{\xi}}

\newcommand{\bzero}{\bm{0}}

\newcommand{\rhoi}{\rho_{\text{i}}}


\begin{document}
\title[Multigrid for glacier modeling]{Multigrid for glacier modeling: \\ A user's guide}

\author{Ed Bueler}

\begin{abstract} FIXME: two principles in introduction: mass conservation complementarity, solver optimality.  four examples in sections \ref{sec:subspace}--\ref{sec:stokes}: poisson equation from subspace decomp point of view, obstacle problem by subset decomposition, monotone multigrid for implicitly-evolving SIA geometry, Schur-complement and Vanka Newton-multigrid for fixed-geometry Glen-Stokes
\end{abstract}

\maketitle

\thispagestyle{empty}
\bigskip

\section{Introduction} \label{sec:intro}

The construction of effective numerical glacier and ice sheet models is challenging for two fundamental reasons.  First, the physics of glaciers is nonlinear, subject to incompletely-understood boundary conditions, and involves a two-phase fluid.  In fact the physics is highly-coupled in the sense that mass, momentum, and energy conservation interact in ways that are both relevant to glaciological modeling goals and which are not well-understood in the literature.  Second, the geometry of glaciers and ice sheets is complex, and in particular the fastest-flowing parts of ice sheets are near or on the geometrically nontrivial lateral boundary.  Numerical models need to perform expensive fine-mesh calculations to accomodate the complicated geometry of ice sheet and glacier boundaries.

On the other hand, since the 1980s researchers in numerial methods have developed multigrid methods to solve partial differential equations like those which describe the ice fluid in glaciers.   For simpler problems like scalar elliptic equations and the linear Stokes system, these methods are now in routine use \cite{Briggsetal2000,Bueler2021,Trottenbergetal2001}.

FIXME write it


\section{The subspace decomposition perspective on multigrid} \label{sec:subspace}

To sketch out our approach to multigrid methods we solve a simple ordinary differential equation (ODE) on an interval, namely the Poisson problem
\begin{equation}
- u''(x) = f(x) \quad \text{on} \quad 0 \le x \le 1. \label{eq:poisson}
\end{equation}
The unknown solution $u(x)$ also satisfies boundary conditions $u(0)=u(1)=0$.  The numerical representation of this problem might use an unequally-spaced mesh of subintervals, called \emph{elements} from now on, as in Figure \ref{fig:finehats}.

\begin{figure}
UNEQUALLY SPACED GRID WITH 12 ELEMENTS (11 INTERIOR POINTS) AND ONE HAT FCN PER POINT
\caption{foo}
\label{fig:finehats}
\end{figure}

The solution $u(x)$ will be built as a linear combination of the piecewise-linear hat functions $\lambda_p(x)$ shown in the Figure, over all interior \emph{nodes} (points) in the mesh:
\begin{equation}
u(x) = \sum_{p=1}^m u^p \lambda_p(x). \label{eq:trialsolution}
\end{equation}
Note that $u(x)$ is continuous and piecewise-linear, and its derivative is defined on the elements, but not generally at the nodes.  Having represented the solution it remains to adjust the coefficients $\{u^p\}$ so as to solve the problem.

Direct use of \eqref{eq:poisson} is possible, thereby generating a finite difference (FD) method, but application to the glacier context will be much clearer if we use a finite element (FE) approach based on re-phrasing \eqref{eq:poisson} in \emph{weak form}.  As most readers will recall, this corresponds to multiplying the equation by a \emph{test function} and integrating by parts so that only first derivatives appear.  Without committing to details we suppose $u$ comes from a vector space of functions $\mathcal{H}$ which are smooth enough for the computations which follow.  Supposing the test function $v$ is in $\mathcal{H}$ also, equation \eqref{eq:poisson} implies
\begin{equation}
\int_0^1 u'(x) v'(x)\,dx = \int_0^1 f(x) v(x)\, dx.
\label{eq:weakpoisson}
\end{equation}
Any FE method for this problem is a discretization which leads from \eqref{eq:weakpoisson} to a linear system
\begin{equation}
A \bu = \bbf \label{eq:linearsystem}
\end{equation}
by substitution of the \emph{trial} formula \eqref{eq:trialsolution} into \eqref{eq:weakpoisson}.  Each hat function is used as a test function, and substitution of $v=\lambda_p$ gives the $p$th equation in system \eqref{eq:linearsystem}.

On this simple basis we can propose a \emph{multilevel} scheme which takes us most of the way to multigrid.  Consider the enlarged set of hat functions in Figure \ref{fig:multilevelhats}

\begin{figure}
SAME GRID AS \eqref{fig:finehats} WITH ORIGNAL FINE HATS DOTTED, 6 ELEMENT MID DASHED, AND 3 ELEMENT COARSE SOLID
\caption{foo}
\label{fig:multilevelhats}
\end{figure}

cite for subspace decomp \cite{Xu1992}

think of residual as a functional; it is represented as a vector in the fine-grid hats but it can act on the coarse grid ones too


\section{Subset decomposition for obstacle problems} \label{sec:obstacle}


cite for multigrid obstacle \cite{BrandtCryer1983,Bueler2021,GraeserKornhuber2009,Jouvetetal2013}; cite for subset decomp \cite{Tai2003}

\section{Multigrid solutions of a shallow ice sheet mass conservation problem} \label{sec:sia}

cite for glaciers as obstacle problems \cite{Bueler2016,Bueler2020,Calvoetal2002,JouvetBueler2012}

\section{Multigrid solutions of a Glen-Stokes momentum conversation problem for a glacier flowline} \label{sec:stokes}

multigrid already used for Blatter-Pattyn model \cite{BrownSmithAhmadia2013} and for hybrid \cite{Jouvetetal2013}; one goal of this section is to make these approaches more understandable; use Schur complement \cite{Bueler2021,Elmanetal2014}; compare Vanka monolithic smoother \cite{Farrelletal2019}

\small

\bigskip
\bibliography{review}
\bibliographystyle{siam}

\end{document}
